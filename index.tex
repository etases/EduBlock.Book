% Options for packages loaded elsewhere
\PassOptionsToPackage{unicode}{hyperref}
\PassOptionsToPackage{hyphens}{url}
\PassOptionsToPackage{dvipsnames,svgnames,x11names}{xcolor}
%
\documentclass[
  letterpaper,
  DIV=11,
  numbers=noendperiod]{scrreprt}

\usepackage{amsmath,amssymb}
\usepackage{lmodern}
\usepackage{iftex}
\ifPDFTeX
  \usepackage[T1]{fontenc}
  \usepackage[utf8]{inputenc}
  \usepackage{textcomp} % provide euro and other symbols
\else % if luatex or xetex
  \usepackage{unicode-math}
  \defaultfontfeatures{Scale=MatchLowercase}
  \defaultfontfeatures[\rmfamily]{Ligatures=TeX,Scale=1}
\fi
% Use upquote if available, for straight quotes in verbatim environments
\IfFileExists{upquote.sty}{\usepackage{upquote}}{}
\IfFileExists{microtype.sty}{% use microtype if available
  \usepackage[]{microtype}
  \UseMicrotypeSet[protrusion]{basicmath} % disable protrusion for tt fonts
}{}
\makeatletter
\@ifundefined{KOMAClassName}{% if non-KOMA class
  \IfFileExists{parskip.sty}{%
    \usepackage{parskip}
  }{% else
    \setlength{\parindent}{0pt}
    \setlength{\parskip}{6pt plus 2pt minus 1pt}}
}{% if KOMA class
  \KOMAoptions{parskip=half}}
\makeatother
\usepackage{xcolor}
\setlength{\emergencystretch}{3em} % prevent overfull lines
\setcounter{secnumdepth}{5}
% Make \paragraph and \subparagraph free-standing
\ifx\paragraph\undefined\else
  \let\oldparagraph\paragraph
  \renewcommand{\paragraph}[1]{\oldparagraph{#1}\mbox{}}
\fi
\ifx\subparagraph\undefined\else
  \let\oldsubparagraph\subparagraph
  \renewcommand{\subparagraph}[1]{\oldsubparagraph{#1}\mbox{}}
\fi


\providecommand{\tightlist}{%
  \setlength{\itemsep}{0pt}\setlength{\parskip}{0pt}}\usepackage{longtable,booktabs,array}
\usepackage{calc} % for calculating minipage widths
% Correct order of tables after \paragraph or \subparagraph
\usepackage{etoolbox}
\makeatletter
\patchcmd\longtable{\par}{\if@noskipsec\mbox{}\fi\par}{}{}
\makeatother
% Allow footnotes in longtable head/foot
\IfFileExists{footnotehyper.sty}{\usepackage{footnotehyper}}{\usepackage{footnote}}
\makesavenoteenv{longtable}
\usepackage{graphicx}
\makeatletter
\def\maxwidth{\ifdim\Gin@nat@width>\linewidth\linewidth\else\Gin@nat@width\fi}
\def\maxheight{\ifdim\Gin@nat@height>\textheight\textheight\else\Gin@nat@height\fi}
\makeatother
% Scale images if necessary, so that they will not overflow the page
% margins by default, and it is still possible to overwrite the defaults
% using explicit options in \includegraphics[width, height, ...]{}
\setkeys{Gin}{width=\maxwidth,height=\maxheight,keepaspectratio}
% Set default figure placement to htbp
\makeatletter
\def\fps@figure{htbp}
\makeatother
\newlength{\cslhangindent}
\setlength{\cslhangindent}{1.5em}
\newlength{\csllabelwidth}
\setlength{\csllabelwidth}{3em}
\newlength{\cslentryspacingunit} % times entry-spacing
\setlength{\cslentryspacingunit}{\parskip}
\newenvironment{CSLReferences}[2] % #1 hanging-ident, #2 entry spacing
 {% don't indent paragraphs
  \setlength{\parindent}{0pt}
  % turn on hanging indent if param 1 is 1
  \ifodd #1
  \let\oldpar\par
  \def\par{\hangindent=\cslhangindent\oldpar}
  \fi
  % set entry spacing
  \setlength{\parskip}{#2\cslentryspacingunit}
 }%
 {}
\usepackage{calc}
\newcommand{\CSLBlock}[1]{#1\hfill\break}
\newcommand{\CSLLeftMargin}[1]{\parbox[t]{\csllabelwidth}{#1}}
\newcommand{\CSLRightInline}[1]{\parbox[t]{\linewidth - \csllabelwidth}{#1}\break}
\newcommand{\CSLIndent}[1]{\hspace{\cslhangindent}#1}

\KOMAoption{captions}{tableheading}
\makeatletter
\makeatother
\makeatletter
\@ifpackageloaded{bookmark}{}{\usepackage{bookmark}}
\makeatother
\makeatletter
\@ifpackageloaded{caption}{}{\usepackage{caption}}
\AtBeginDocument{%
\ifdefined\contentsname
  \renewcommand*\contentsname{Table of contents}
\else
  \newcommand\contentsname{Table of contents}
\fi
\ifdefined\listfigurename
  \renewcommand*\listfigurename{List of Figures}
\else
  \newcommand\listfigurename{List of Figures}
\fi
\ifdefined\listtablename
  \renewcommand*\listtablename{List of Tables}
\else
  \newcommand\listtablename{List of Tables}
\fi
\ifdefined\figurename
  \renewcommand*\figurename{Figure}
\else
  \newcommand\figurename{Figure}
\fi
\ifdefined\tablename
  \renewcommand*\tablename{Table}
\else
  \newcommand\tablename{Table}
\fi
}
\@ifpackageloaded{float}{}{\usepackage{float}}
\floatstyle{ruled}
\@ifundefined{c@chapter}{\newfloat{codelisting}{h}{lop}}{\newfloat{codelisting}{h}{lop}[chapter]}
\floatname{codelisting}{Listing}
\newcommand*\listoflistings{\listof{codelisting}{List of Listings}}
\makeatother
\makeatletter
\@ifpackageloaded{caption}{}{\usepackage{caption}}
\@ifpackageloaded{subcaption}{}{\usepackage{subcaption}}
\makeatother
\makeatletter
\@ifpackageloaded{tcolorbox}{}{\usepackage[many]{tcolorbox}}
\makeatother
\makeatletter
\@ifundefined{shadecolor}{\definecolor{shadecolor}{rgb}{.97, .97, .97}}
\makeatother
\makeatletter
\makeatother
\ifLuaTeX
  \usepackage{selnolig}  % disable illegal ligatures
\fi
\IfFileExists{bookmark.sty}{\usepackage{bookmark}}{\usepackage{hyperref}}
\IfFileExists{xurl.sty}{\usepackage{xurl}}{} % add URL line breaks if available
\urlstyle{same} % disable monospaced font for URLs
\hypersetup{
  pdftitle={EduBlock},
  pdfauthor={HSGamer},
  colorlinks=true,
  linkcolor={blue},
  filecolor={Maroon},
  citecolor={Blue},
  urlcolor={Blue},
  pdfcreator={LaTeX via pandoc}}

\title{EduBlock}
\author{HSGamer}
\date{9/20/2022}

\begin{document}
\maketitle
\ifdefined\Shaded\renewenvironment{Shaded}{\begin{tcolorbox}[breakable, borderline west={3pt}{0pt}{shadecolor}, enhanced, sharp corners, frame hidden, interior hidden, boxrule=0pt]}{\end{tcolorbox}}\fi

\renewcommand*\contentsname{Table of contents}
{
\hypersetup{linkcolor=}
\setcounter{tocdepth}{2}
\tableofcontents
}
\bookmarksetup{startatroot}

\hypertarget{preface}{%
\chapter*{Preface}\label{preface}}
\addcontentsline{toc}{chapter}{Preface}

This is a Quarto book.

To learn more about Quarto books visit
\url{https://quarto.org/docs/books}.

\bookmarksetup{startatroot}

\hypertarget{edublock---kux1ebf-houx1ea1ch}{%
\chapter{EduBlock - Kế hoạch}\label{edublock---kux1ebf-houx1ea1ch}}

\hypertarget{phase-1---khux1edfi-ux111ux1ed9ng}{%
\section{Phase 1 - Khởi động}\label{phase-1---khux1edfi-ux111ux1ed9ng}}

\textbf{Thời gian}: 01/08/2022 - 04/09/2022

\textbf{Mô tả}: Đây là giai đoạn lên ý tưởng và thực hiện prototype đầu
tiên cho dự án Học bạ điện tử, với phạm vi của giai đoạn là thực hiện hệ
thống trên 1 trường, và thử nghiệm trên nền tảng public blockchain
(Dfinity).

\textbf{Sản phẩm}: - Tài liệu yêu cầu - Tài liệu giới thiệu dự án - Tài
liệu cấu trúc hệ thống Phase 1 - Hi-Fi Prototype Phase 1

\hypertarget{phase-2---vux1b0ux1ee3t-chux1b0ux1edbng-ngux1ea1i-vux1eadt}{%
\section{Phase 2 - Vượt chướng ngại
vật}\label{phase-2---vux1b0ux1ee3t-chux1b0ux1edbng-ngux1ea1i-vux1eadt}}

\textbf{Thời gian}: 05/09/2022 - 30/09/2022

\textbf{Mô tả}: Đây là giai đoạn quan trọng để mở rộng phạm vi của dự án
lên quy mô toàn quốc, bao gồm nghiên cứu các công nghệ cần cho việc
thiết kế hệ thống phù hợp với quy trình xét học bạ hiện tại.

\hypertarget{task-1---mux1edf-rux1ed9ng-phux1ea1m-vi}{%
\subsection{Task 1 - Mở rộng phạm
vi}\label{task-1---mux1edf-rux1ed9ng-phux1ea1m-vi}}

\textbf{Mô tả}: Mở rộng phạm vi dự án và xây dựng cái tài liệu mới dựa
trên các tài liệu từ Phase 1. Chủ yếu là thiết kế mô hình hệ thống cho
phạm vi được mở rộng này.

\textbf{Lưu ý}: Phạm vi lần này là mở rộng quy mô lên các Sở Giáo Dục
của toàn quốc, và phạm vi học bạ bao gồm các điểm số nhỏ (Kiểm tra
miệng, Kiểm tra 15 phút,\ldots).

\textbf{Sản phẩm}: - Tài liệu yêu cầu - Tài liệu giới thiệu dự án theo
phạm vi mới - Tài liệu cấu trúc hệ thống Phase 2

\hypertarget{task-2---thiux1ebft-kux1ebf-backend}{%
\subsection{Task 2 - Thiết kế
Backend}\label{task-2---thiux1ebft-kux1ebf-backend}}

\textbf{Mô tả}: Thiết kế hệ thống backend theo phạm vi mới. Tìm hiểu
công nghệ Hyperledger Fabric để xây dựng mạng lưới Private Blockchain
phù hợp với yêu cầu.

\textbf{Lưu ý}: Nếu không kịp tìm hiểu Hyperledger Fabric hoặc nó không
phù hợp thì vẫn có thể thiết kế hệ thống backend dạng MVC với phạm vi
của từng Sở Giáo Dục và tìm cách liên kết các Sở Giáo Dục lại thành mạng
lưới truyền thông tin.

\textbf{Sản phẩm} - Tài liệu thiết kế Backend - Tài liệu Endpoint của
Web API - Hi-Fi Prototype Phase 2 cho Backend

\hypertarget{task-3---thiux1ebft-kux1ebf-frontend}{%
\subsection{Task 3 - Thiết kế
Frontend}\label{task-3---thiux1ebft-kux1ebf-frontend}}

\textbf{Mô tả}: Thiết kế Backend sử dụng Framework phù hợp cho người lập
trình. Mục tiêu là sử dụng Framework để vừa làm nhanh, phù hợp với người
lập trình web thuần HTML, CSS và JS mà không có các logic phức tạp và
lỗi giao diện.

\textbf{Lưu ý}: Có thể tự thiết kế Fake Web API để thử nghiệm việc kết
nối đến Backend.

\textbf{Ghi chú}: Svelte?

\textbf{Sản phẩm} - Tài liệu thiết kế Frontend - Hi-Fi Prototype Phase 2
cho Frontend

\hypertarget{task-4---mux1edf-rux1ed9ng-tuxednh-nux103ng-nuxe2ng-cao}{%
\subsection{Task 4 - Mở rộng tính năng nâng
cao}\label{task-4---mux1edf-rux1ed9ng-tuxednh-nux103ng-nuxe2ng-cao}}

\textbf{Mô tả}: Bao gồm sử dụng công nghệ nhận diện để nạp học bạ giấy
lên hệ thống.

\textbf{Lưu ý}: Ở Phase này là nghiên cứu tìm cách thêm tính năng, những
tính năng trong Task này nếu xét là không kịp thì có thể không làm.

\textbf{Sản phẩm} - Tài liệu và Prototype liên quan

\hypertarget{phase-3---tux103ng-tux1ed1c}{%
\section{Phase 3 - Tăng tốc}\label{phase-3---tux103ng-tux1ed1c}}

\textbf{Thời gian}: 01/10/2022 - 31/10/2022

\textbf{Mô tả}: Đây là giai đoạn thực hiện sản phẩm cuối cùng. Bao gồm
hoàn thiện Frontend, Backend và các tính năng liên quan; Kết nối
Frontend và Backend; Bổ sung tài liệu cài đặt hệ thống.

\textbf{Sản phẩm} - Hệ thống cuối, chưa qua Test - Tài liệu hệ thống -
Hướng dẫn cài đặt hệ thống

\hypertarget{phase-4---vux1ec1-ux111uxedch}{%
\section{Phase 4 - Về đích}\label{phase-4---vux1ec1-ux111uxedch}}

\textbf{Thời gian}: 01/11/2022 - 30/11/2022

\textbf{Mô tả}: Đây là giai đoạn test hệ thống và viết báo cáo

\textbf{Sản phẩm} - Hệ thống cuối, đã Test - Tài liệu Testing - Tài liệu
hệ thống cuối - Báo cáo dự án

\bookmarksetup{startatroot}

\hypertarget{edublock}{%
\chapter{EduBlock}\label{edublock}}

\hypertarget{introduction}{%
\section{Introduction}\label{introduction}}

This is the specification for a student record system that store the
student records on the BlockChain network. The system is designed to
provide a user-friendly interface for teachers to update their students'
academic records and a full replacement of traditional academic record
papers.

\hypertarget{about-existing-system}{%
\section{About Existing System}\label{about-existing-system}}

Many high schools in Viet Nam use one of existing online systems to
store students' academic records and notify their parents about recent
records, but they still use traditional record papers for
post-graduation \& university / college enrollment. Therefore, The
teachers find it difficult to update their students' records because
they have to update the records on both the online system \& the record
paper. Moreover, the online system is a centralized system that only the
admins and the teachers can see and interact, so there is little
transparency for students who want to see their records at any time.

\hypertarget{user-requirement}{%
\section{User Requirement}\label{user-requirement}}

\hypertarget{main-function}{%
\subsection{Main Function}\label{main-function}}

\begin{enumerate}
\def\labelenumi{\arabic{enumi}.}
\tightlist
\item
  Normal User

  \begin{itemize}
  \tightlist
  \item
    Request Registration

    \begin{itemize}
    \tightlist
    \item
      The user must fill a form of personal information
    \item
      The user must choose a role to request: Student or Teacher
    \item
      The user can send the request form and wait for an admin to review
    \end{itemize}
  \item
    Login

    \begin{itemize}
    \tightlist
    \item
      The user must have an account on the system to login
    \end{itemize}
  \end{itemize}
\item
  Student

  \begin{itemize}
  \tightlist
  \item
    See Student Records

    \begin{itemize}
    \tightlist
    \item
      The user must login to use this function
    \item
      The user can only see his academic records
    \end{itemize}
  \end{itemize}
\item
  Teacher

  \begin{itemize}
  \tightlist
  \item
    View Class

    \begin{itemize}
    \tightlist
    \item
      The user must login to use this function
    \item
      The user can only see his own classes
    \item
      The user can filter his classes by name or year
    \end{itemize}
  \item
    View Student

    \begin{itemize}
    \tightlist
    \item
      The user must login to use this function
    \item
      The user can only see his own classes
    \item
      The user can choose a class and see its student
    \item
      The user can filter his students by name
    \end{itemize}
  \item
    Update Student Records

    \begin{itemize}
    \tightlist
    \item
      The user must login to use this function
    \item
      The user can only update students of his classes
    \item
      The user can choose a student and see his academic records
    \item
      The user can update the student's records
    \end{itemize}
  \end{itemize}
\item
  Admin

  \begin{itemize}
  \tightlist
  \item
    View Class

    \begin{itemize}
    \tightlist
    \item
      The user must login to use this function
    \item
      The user can see created classes
    \item
      The user can filter the classes by name or year
    \end{itemize}
  \item
    Create Class

    \begin{itemize}
    \tightlist
    \item
      The user must login to use this function
    \item
      The user must assign a teacher to the class as a home teacher
    \item
      The user must assign students to the class
    \end{itemize}
  \item
    View Student

    \begin{itemize}
    \tightlist
    \item
      The user must login to use this function
    \item
      The user can see registered students
    \item
      The user can filter the students by name
    \end{itemize}
  \item
    View Teacher

    \begin{itemize}
    \tightlist
    \item
      The user must login to use this function
    \item
      The user can see registered teachers
    \item
      The user can filter the teachers by name
    \end{itemize}
  \item
    Review Registration

    \begin{itemize}
    \tightlist
    \item
      The user must login to use this function
    \item
      The user can see the waiting requests
    \item
      The user can accept or deny a request
    \item
      The user can filter the requests by username or fullname
    \item
      For teacher requests, the user can export a list to give to the
      On-Chain Admin
    \end{itemize}
  \end{itemize}
\item
  On-Chain Admin

  \begin{itemize}
  \tightlist
  \item
    Add/Remove Teachers

    \begin{itemize}
    \tightlist
    \item
      The user must interact with the BlockChain (On-Chain) service to
      use this function
    \item
      The user can add or remove a list of teachers received from the
      Admin
    \end{itemize}
  \end{itemize}
\end{enumerate}

\hypertarget{non-function}{%
\subsection{Non-Function}\label{non-function}}

\begin{itemize}
\tightlist
\item
  The system easy to maintain \& upgrade
\item
  The user interface is clear, idiot-proof, easy to use \& friendly
\item
  The system is available on 24/7
\end{itemize}

\hypertarget{system-requirement}{%
\section{System Requirement}\label{system-requirement}}

\hypertarget{system-component}{%
\subsection{System Component}\label{system-component}}

\begin{figure}

{\centering \includegraphics[width=0.5\textwidth,height=\textheight]{index_files/mediabag/ce96683be2cc487054d9b59b1e9414d18a599b70.svg}

}

\caption{Overview of System component}

\end{figure}

\begin{longtable}[]{@{}
  >{\raggedright\arraybackslash}p{(\columnwidth - 2\tabcolsep) * \real{0.5000}}
  >{\raggedright\arraybackslash}p{(\columnwidth - 2\tabcolsep) * \real{0.5000}}@{}}
\toprule()
\begin{minipage}[b]{\linewidth}\raggedright
Component
\end{minipage} & \begin{minipage}[b]{\linewidth}\raggedright
Description
\end{minipage} \\
\midrule()
\endhead
Frontend & The interface of the system, responsible for UI/UX \\
Off-Chain Service & Store the personal information of the user, the
registration requests and the details of classes \& students \\
On-Chain Service & Store the student's academic records by grade \&
Allow teachers to update their students' records \\
\bottomrule()
\end{longtable}

\hypertarget{actor-description}{%
\subsection{Actor Description}\label{actor-description}}

\begin{longtable}[]{@{}
  >{\raggedright\arraybackslash}p{(\columnwidth - 2\tabcolsep) * \real{0.5000}}
  >{\raggedright\arraybackslash}p{(\columnwidth - 2\tabcolsep) * \real{0.5000}}@{}}
\toprule()
\begin{minipage}[b]{\linewidth}\raggedright
Actor
\end{minipage} & \begin{minipage}[b]{\linewidth}\raggedright
Description
\end{minipage} \\
\midrule()
\endhead
Student & The smallest actor of the system who can only see his academic
records \\
Teacher & A person who can update the student's academic records \\
Admin & A manager of the system who manage teachers, classes \&
students \\
On-Chain Admin & A subset of Admin who interacts with the On-Chain
service \\
\bottomrule()
\end{longtable}

\hypertarget{use-case-diagram}{%
\subsection{Use case diagram}\label{use-case-diagram}}

\begin{figure}

{\centering \includegraphics[width=0.5\textwidth,height=\textheight]{index_files/mediabag/131b9aa307355063fbfb72c9d1de5c3fb86ee1a5.svg}

}

\caption{Use case diagram}

\end{figure}

\bookmarksetup{startatroot}

\hypertarget{srs}{%
\chapter{SRS}\label{srs}}

\hypertarget{system-architecture}{%
\section{System Architecture}\label{system-architecture}}

\begin{itemize}
\tightlist
\item
  Without BlockChain
\end{itemize}

\begin{figure}

{\centering \includegraphics[width=0.5\textwidth,height=\textheight]{index_files/mediabag/e6b425c0fbe505aecc3ee02c806d60f2bf722007.svg}

}

\caption{System architecture without BlockChain}

\end{figure}

\begin{itemize}
\tightlist
\item
  With BlockChain
\end{itemize}

\begin{figure}

{\centering \includegraphics[width=0.5\textwidth,height=\textheight]{index_files/mediabag/7572a3fcc4bfb3985f2f126d82ba1aeba64bc877.svg}

}

\caption{System architecture with BlockChain}

\end{figure}

\begin{longtable}[]{@{}
  >{\raggedright\arraybackslash}p{(\columnwidth - 2\tabcolsep) * \real{0.0588}}
  >{\raggedright\arraybackslash}p{(\columnwidth - 2\tabcolsep) * \real{0.9412}}@{}}
\toprule()
\begin{minipage}[b]{\linewidth}\raggedright
Component
\end{minipage} & \begin{minipage}[b]{\linewidth}\raggedright
Description
\end{minipage} \\
\midrule()
\endhead
Chain Node (CN) & A node of the blockchain. This stores the records and
handles the history and transaction requests from the Request Server
(Change/View the score, information, etc.) \\
Trusted Record Service (TRS) & Similar to Chain Node, but this is a
centralized \& trusted (by all nodes) service that stores the records
and handles requests from the Request Server \\
Request Server & The off-chain backend of a CN / TRS. This stores the
pending requests from the user and is the only way to call a request to
the CN / TRS. Each Request Server may have a different way to handle
user requests (Voting, Direct Request, etc.) \\
Frontend Server & Provide the UX/UI for interacting with the Request
Server \\
\bottomrule()
\end{longtable}

\bookmarksetup{startatroot}

\hypertarget{references}{%
\chapter*{References}\label{references}}
\addcontentsline{toc}{chapter}{References}

\hypertarget{refs}{}
\begin{CSLReferences}{0}{0}
\end{CSLReferences}



\end{document}
